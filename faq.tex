%%%%%%%%%%%%%%%%%%%%%%%%%%%%%%%%%%%%%%%%%
% Frequently Asked Questions
% LaTeX Template
% Version 1.0 (22/7/13)
%
% This template has been downloaded from:
% http://www.LaTeXTemplates.com
%
% Original author:
% Adam Glesser (adamglesser@gmail.com)
%
% License:
% CC BY-NC-SA 3.0 (http://creativecommons.org/licenses/by-nc-sa/3.0/)
%
%%%%%%%%%%%%%%%%%%%%%%%%%%%%%%%%%%%%%%%%%

\documentclass[11pt]{article}

\usepackage[utf8]{inputenc}
\usepackage[margin=1in]{geometry} % Required to make the margins smaller to fit more content on each page
\usepackage[linkcolor=blue]{hyperref} % Required to create hyperlinks to questions from elsewhere in the document
\hypersetup{pdfborder={0 0 0}, colorlinks=true, urlcolor=blue} % Specify a color for hyperlinks
\usepackage{todonotes} % Required for the boxes that questions appear in
\usepackage{tocloft} % Required to give customize the table of contents to display questions
\usepackage{microtype} % Slightly tweak font spacing for aesthetics
\usepackage{palatino} % Use the Palatino font

\setlength\parindent{0pt} % Removes all indentation from paragraphs

% Create and define the list of questions
\newlistof{questions}{faq}{\large Elenco delle attività svolte} % This creates a new table of contents-like environment that will output a file with extension .faq
\setlength\cftbeforefaqtitleskip{4em} % Adjusts the vertical space between the title and subtitle
\setlength\cftafterfaqtitleskip{1em} % Adjusts the vertical space between the subtitle and the first question
\setlength\cftparskip{.3em} % Adjusts the vertical space between questions in the list of questions

% Create the command used for questions
\newcommand{\question}[1] % This is what you will use to create a new question
{
\refstepcounter{questions} % Increases the questions counter, this can be referenced anywhere with \thequestions
\par\noindent % Creates a new unindented paragraph
\phantomsection % Needed for hyperref compatibility with the \addcontensline command
\addcontentsline{faq}{questions}{\thequestions - #1} % Adds the question to the list of questions
\todo[inline, color=gray!40]{\thequestions - \textbf{#1}} % Uses the todonotes package to create a fancy box to put the question
\vspace{1em} % White space after the question before the start of the answer
}

% Uncomment the line below to get rid of the trailing dots in the table of contents
% \renewcommand{\cftdot}{}

% Uncomment the two lines below to get rid of the numbers in the table of contents
%\let\Contentsline\contentsline
%\renewcommand\contentsline[3]{\Contentsline{#1}{#2}{}}

\begin{document}

%----------------------------------------------------------------------------------------
%	TITLE AND LIST OF QUESTIONS
%----------------------------------------------------------------------------------------

\begin{center}
\Huge{\bf \emph{Relazione sull'attività di tutoraggio per l'orientamento}} % Main title
\end{center}

\begin{center}
	\huge Buracchi Marco - Matricola 6011920
\end{center}

\listofquestions % This prints the subtitle and a list of all of your questions

\newpage % Comment this if you would like your questions and answers to start immediately after table of questions

%----------------------------------------------------------------------------------------
%	Elenco
%----------------------------------------------------------------------------------------

\question{Incontri conoscitivi}\label{uno}
	\begin{itemize}
		\item Il giorno 19/02 ho effettuato l'incontro conoscitivo con Patrizia Maranghi e gli altri tutor nel quale abbiamo discusso e organizzato le prime attività da svolgere. Come primo impegno ci sarà il controllo e il miglioramento dei siti dei vari corsi di laurea. 
		
		Questo incontro è durato 1 ora.
		
		\emph{ORE CUMULATIVE:} 1
	\end{itemize}
	
\question{Riunione organizzativa}\label{due}
	\begin{itemize}
		\item Il giorno 27/02 ho partecipato alla riunione organizzativa nel quale sono stati assegnati gli incarichi ai vari tutor. Personalmente sono stato incarico di revisionare il sito istituzionale del corso di fisica e di fungere da supervisore per gli altri tutor addetti alla revisione dei siti per comunicare direttamente con Adriana Ardy.
		
		Questa attività mi ha impegnato per 2 ore.
		
		\emph{ORE CUMULATIVE:} 3
	\end{itemize}
	
\question{Doodle sportello}\label{tre}
	\begin{itemize}
		\item Il giorno 27/02 ho creato un doodle dedicato all'organizzazione delle attività di sportello per l'orientamento. Con questo doodle abbiamo deciso i turni da effettuare per prestare il nostro servizio presso lo sportello per l'orientamento nei mesi che vanno da Giugno ad Ottobre.
		
		Questo lavoro è durato 1 ora.
		
		\emph{ORE CUMULATIVE:} 4:00
	\end{itemize}
	
\question{Siti istituzionali}\label{quattro}
	\begin{itemize}
		\item Il giorno 28/02 ho iniziato il controllo del sito istituzionale del corso triennale di fisica. Ho rilevato le criticità segnalate nel documento consegnato ad Adriana.
		
		Questo lavoro è durato 2 ore.
		
		\emph{ORE CUMULATIVE:} 6:00		
		\item Il giorno 05/03 ho iniziato il controllo del sito istituzionale del corso magistrale di fisica. Ho rilevato le criticità segnalate nel documento consegnato ad Adriana.
		
		Questo lavoro è durato 2 ore.
		
		\emph{ORE CUMULATIVE:} 8		
		\item In generale il lavoro di "supervisore" mi ha impegnato per ulteriori 4 ore con attività di comunicazione sia con Adriana che con gli altri tutor e con la raccolta e l'unificazione dei vari lavori in un singolo file che ho consegnato ad Adriana.
		
		\emph{ORE CUMULATIVE:} 12:00		
		\item Il giorno 20/03 ho partecipato alla riunione con i webmaster inerente alle problematiche riscontrate sui siti istituzionali.
		
		L'incontro è durato 3 ore.
		
		\emph{ORE CUMULATIVE:} 15:00
		\item Il giorno 27/03 ho partecipato alla riunione con la professoressa Verri e il dottor Ceccarelli per discutere delle problematiche riscontrate nel sito di informatica. 
		
		L'incontro è durato 1 ora.
		
		\emph{ORE CUMULATIVE:} 16:00
		\item Il giorno 29/03 ho partecipato alla riunione con Adriana Ardy e gli altri tutor nella quale ci è stato mostrato il sistema di modifica dei siti istituzionali. 
		
		L'incontro è durato 3 ore.
		
		\emph{ORE CUMULATIVE:} 19:00
		\item Il giorno 03/04, su richiesta del dott. Maggesi e della prof.ssa Verri, ho effettuato un lavoro di controllo sui dati inseriti nei syllabus dei corsi di matematica ed informatica.
		
		Questo lavoro è durato 4 ore.
		
		\emph{ORE CUMULATIVE:} 23:00
		\item Il giorno 04/04, su richiesta del dott. Maggesi, ho informato tutti i professori del corso di laurea di matematica sulle sezioni mancanti del Syllabus inerenti ai propri corsi.
		
		Questo lavoro è durato 3 ore.
		
		\emph{ORE CUMULATIVE:} 26:00
		\item Il giorno 05/04, su richiesta della professoressa Verri, ho informato tutti i professori del corso di laurea di informatica sulle sezioni mancanti del Syllabus inerenti ai propri corsi.
		
		Questo lavoro è durato 3 ore.
		
		\emph{ORE CUMULATIVE:} 29:00
		
		\item Il giorno 12/06, su richiesta della professoressa Verri, ho controllato tutte le pagine dei corsi del sito della magistrale di informatica verificando che i campi \emph{"obiettivi formativi"} e \emph{"modalità di verifica dell'apprendimento"} contenessero informazioni complete ed esaurienti.
		
		Questo lavoro è durato 6 ore.
		
		\emph{ORE CUMULATIVE:} 35:00
	\end{itemize}
	
\question{PF24}\label{cinque}
	\begin{itemize}
		\item Il giorno 19/03 ho partecipato all'incontro formativo per effettuare attività di sorveglianza agli esami relativi ai PF24.
		
		L'incontro è durato 2 ore.
		
		\emph{ORE CUMULATIVE:} 37:00
		\item Il giorno 23/03 e 24/03 ho partecipato all'attività di sorveglianza agli esami PF24.
		
		Sono stato impegnato in tutto per 14 ore.
		
		\emph{ORE CUMULATIVE:} 51:00
	\end{itemize}

\question{Un giorno all'università}\label{sei}
	\begin{itemize}
		\item Il giorno 14/04 ho partecipato all'evento di orientamento "un giorno all'università" aiutando la professoressa Verri allo stand del CdL di informatica. 
		
		Sono stato impegnato per 6 ore.
		
		\emph{ORE CUMULATIVE:} 57:00
	\end{itemize}

\question{L'impresa si presenta}\label{sei}
	\begin{itemize}
		\item Il giorno 22/05 ho partecipato all'evento di orientamento "L'impresa si presenta" aiutando nell'organizzazione, nella raccolta firme per presenza e nella distribuzione del materiale informativo. 
		
		Sono stato impegnato per 6 ore.
		
		\emph{ORE CUMULATIVE:} 63:00
		
		\item Il giorno 23/05 ho creato un modulo excel per permettere l'analisi dei questionari proposti agli eventi "L'impresa si presenta" ed ho inserito quelli dell'evento del 22/05.
		
		Per questo lavoro sono state necessarie 2 ore.
		
		\emph{ORE CUMULATIVE:} 65:00
		
		\item Il giorno 24/05 ho migliorato il modulo excel precedente inserendo nuovi grafici e un maggior numero di spazi per inserire questionari.
		
		Per questo lavoro sono state necessarie 2 ore.
		
		\emph{ORE CUMULATIVE:} 67:00
	\end{itemize}
	
%%----------------------------------------------------------------------------------------
%%	QUESTIONS AND ANSWERS
%%----------------------------------------------------------------------------------------
%
%\question{How do I add a new question and answer?}\label{new-question}
%
%In the code, type:
%
%\begin{verbatim}
%\question{A question that needs answering}\label{question-label}
%
%The answer to this question.
%\end{verbatim}
%
%%------------------------------------------------
%
%\question{Why is there a label in the code for \hyperref[new-question]{the previous question}?}\label{labels}
%
%This is not necessary, but it does give you a way of linking to a different question. In order to link to another question you simply need to add the following:
%
%\begin{verbatim}
%\hyperref[question-label]{click here}
%\end{verbatim}
%
%The first part \texttt{[question-label]} is the label name and the second part \texttt{\{click here\}} is the text that is displayed as link.
%
%%------------------------------------------------
%
%\question{How do I change the title and subtitle?}\label{change-title}
%
%To change the main title, simply find the "TITLE AND LIST OF QUESTIONS" block and replace "A Template for FAQ's" within it. To change the subtitle find the following command:
%
%\begin{verbatim}
%\newlistof{questions}{faq}{\large List of Frequently Asked Questions}
%\end{verbatim}
%
%and replace the subtitle with one of your choosing.
%
%%------------------------------------------------
%
%\question{Is it possible to change the spacing between the questions in the list of questions?}\label{change-spacing}
%
%Yes, simply find the following line:
%
%\begin{verbatim}
%\setlength\cftparskip{.3em}
%\end{verbatim}
%
%and change the \texttt{.3em} to whatever suits your fancy.
%
%%------------------------------------------------
%
%\question{What if I want to hide the page numbers and/or trailing dots next to the question in the list of questions?}\label{page-numbering}
%
%To remove the trailing dots to the page numbers, find the line:
%
%\begin{verbatim}
%%\renewcommand{\cftdot}{}
%\end{verbatim}
%and uncomment it. To remove the page numbers as well, find the following lines and uncomment them:
%\begin{verbatim}
%%\let\Contentsline\contentsline
%%\renewcommand\contentsline[3]{\Contentsline{#1}{#2}{}}
%\end{verbatim}
%
%%------------------------------------------------
%
%\question{Is it possible to number questions?}\label{number-questions}
%
%Yes, you can refer to the number of the current question with:
%
%\begin{verbatim}
%\thequestions
%\end{verbatim}
% 
%For example, this is question \thequestions. You can even incorporate question numbers into the questions and list of questions automatically by adding:
%
%\begin{verbatim}
%Question \thequestions:
%\end{verbatim}
%
%just before each \texttt{\#1} in the \texttt{\textbackslash questions} definition block in the preamble.
%
%%------------------------------------------------
%
%\question{Question \thequestions: Can I change the color of the question boxes?}\label{question-color}
%
%Just find the following line and change the color specified there:
%
%\begin{verbatim}
%\todo[inline, color=green!40]{\textbf{#1}}
%\end{verbatim}
%
%%----------------------------------------------------------------------------------------

\end{document}